% Created 2019-12-18 Wed 16:15
% Intended LaTeX compiler: pdflatex
\documentclass[a4paper, justified, notitlepage, sfsidenotes, notoc]{tufte-book}
     \input{/users/rakhim/.emacs.d/latex/tufte.tex}
\usepackage[utf8]{inputenc}
\usepackage[T1]{fontenc}
\usepackage{graphicx}
\usepackage{grffile}
\usepackage{longtable}
\usepackage{wrapfig}
\usepackage{rotating}
\usepackage[normalem]{ulem}
\usepackage{amsmath}
\usepackage{textcomp}
\usepackage{amssymb}
\usepackage{capt-of}
\usepackage{hyperref}
\date{}
\title{Computer Science For The Busy Developer}
\hypersetup{
 pdfauthor={Rakhim Davletkaliyev},
 pdftitle={Computer Science For The Busy Developer},
 pdfkeywords={},
 pdfsubject={},
 pdfcreator={Emacs 26.3 (Org mode 9.1.9)},
 pdflang={English}}
\begin{document}

\part{Direct proof}
\label{sec:org243cb8d}

Back when we were discussing set cardinality, we've successfully proved a statement: \emph{If \(A\) is a finite set of \(m\) elements, then there are \(2^{m}\) subsets of \(A\).} Let's talk about proofs in more detail, and discuss different proof techniques.

A proof is a sequence of mathematical statements, a path from some basic truth to the desired outcome. An impeccable argument, if you will. Possible the simplest form of proof is a direct proof. It's a direct attempt to see what the statement means if we dare to play with it. Consider a theorem:

\textbf{Theorem 1.} If \(n\) is an odd positive integer, then \(n^2\) is odd.

\textbf{Proof.} An odd positive integer can be written as \(n = 2k + 1\), since something like \(2k\) is even and adding 1 makes it definitely odd. We're interested in what odd squared looks like, so let's square this definition:

$$ n^2 = (2k + 1)^2 = $$
$$ 4k^2 + 4k + 1 = $$
$$ 2(2k^2 + 2k) + 1 $$

So, we have this final formula \(2(2k^2 + 2k) + 1\) and it follows the pattern of \(2k + 1\). This means it's odd! We have a proof.

This theorem is based on an idea that numbers described as \(2k + 1\) are definitely odd. This might be another theorem that requires another proof, and that proof might be based on some other theorems. The general idea of mathematics is that if you follow any theorem to the very beginning, you'll meet the fundamental axioms, the basis of everything.

Now that we have this proven theorem in our arsenal, let's take a look at another theorem and prove it by contradiction.
\end{document}
